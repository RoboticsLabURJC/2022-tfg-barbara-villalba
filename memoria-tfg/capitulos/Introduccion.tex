\chapter{Introducción}
\label{cap:introduccion}
\setcounter{page}{1}

En la última década, la evolución tecnológica ha provocado una transformación radical en nuestra forma de vivir, trabajar y relacionarnos desempeñando la tecnología un papel 
fundamental en el avance de la sociedad e impulsando una serie de innovaciones que se extienden desde la invención de la rueda hasta la era digital contemporánea. 
Por ejemplo, los ordenadores empezaron siendo grandes máquinas que ocupaban habitaciones enteras que requerían una gran cantidad de energía y mantenimiento. Hoy en día, los ordenadores
son dispositivos ligeros y eficientes que pueden realizar múltiples cálculos por segundos que se utilizan en diferentes ramas de las ingenierías como la informática, telecomunicaciones y,
por supuesto, la robótica. \newline

La robótica,en particular, se destaca como una de las ramas de la tecnología que más impacto significativo ha tenido. Estos avances han facilitado numerosas tareas mejorando la eficiencia 
y la capacidad de enfrentar desafíos complejos dando pie a nuevas posibilidades en nuestro entorno. Dentro de las ramas de la robótica, la robótica aerea con el uso de los drones han 
demostrado ser desafiantes y valiosos en la inspección de áreas de díficil acceso, el mapeo de terrenos, la realización de entregas, la navegación autonónoma o la captura 
de imágenes desde alturas elevadas. Su versatilidad y su capcidad para poder operar en entornos peligrosos o inaccesibles para los seres humanos los convierten 
en herramientas fundamentales en campos como la agricultura, la seguridad, la investigación medioambiental y la logística. 

\section{La robótica}
\label{sec:enfoquesrobotica}
Como mencionamos anteriorente, entre las diversas ramas de la tecnología, la robótica se destaca como una de las más prometedoras. Apareciendo como disciplina durante la década de los años 60, 
la robótica ha tenido un cambio asombroso  pasando de ser simples máquinas programables a sistemas inteligentes capaces de aprender y adaptarse 
a su entorno teniendo avances en diversas disciplinas de la ingeniería, como la informática, la inteligencia artificial, la ingenieria de control, la mecánica y otras más. 
Los robots de hoy en día no solo tienen la capacidad de realizar tareas programadas y repetitivas, sino que también tienen la capacidad de interactuar con su entorno
, tomar decisiones basadas en la información sensorial y aprender de sus experiencias. Este avance en la robótica nos ha permitido tener una definición más precisa de lo que es 
la robótica moderna, definiendo la robótica como ciencia interdisciplinaria encargada de la creación, funcionamiento, estructuración, fabricación y uso de los robots. 
Esta definición mencionada incluye no solo los componentes mecánicos y eléctricos, sino que también los algoritmos que los controlan, los sensores que les permiten recopilar
datos de su entorno y los sistemas que procesan esta información y toman decisiones.

\begin{figure} [H]
  \begin{center}
    \includegraphics[scale=0.25]{figs/introducción/robot.png}
  \end{center}
  \caption{Definición de robot.}
  \label{fig:robot}
\end{figure}\

Lo que hace que un robot tenga la capacidad de aprender y adaptarse a un entorno abierto de nuevas oportunidades para la robótica, como la medicina, la exploración
lunar, la asistencia personal, la automatización industrial y más. Además de abrir nuevas aplicaciones y tareas como puede ser la navegación autónoma, la detención de objetos o 
la manipulación de objetos con sensores táctiles y de fuerza, dichas tareas pueden realizar pueden ser peligrosas, delicadas, sucias o monótonas 
(conocidas como las 4D's: dull,dirty, dangerous and dear) \footnote{:\url{https://www.forbes.com/sites/bernardmarr/2017/10/16/the-4-ds-of-robotization-dull-dirty-dangerous-and-dear/?sh=40bb6cec3e0d}}

 % etiqueta para luego referenciar esta sección
 \subsection{Enfoques en la robótica}
 \label{sec:enfoquesrobotica}
A lo largo de la evolución de la robótica, han surgido tres enfoques fundamentales para el diseño y la operación de robots, cada uno de estos enfoques presentan diferentes
formas de interactuar y operar robots, con sus propias características y aplicaciones únicas. 

\subsubsection{Teleoperación}
\label{sec:subseccion}

La teleoperación surge de la necesidad de manipular objetos o realizar tareas en entornos complejos, peligrosos y distantes para el ser humano. Desde la historia, el ser humano
ha utilizado una variedad de herramientas para ampliar su capacidad de manipulación como palos utilizados para caer la fruta madura de un arbol. Con el tiempo, se desarrollaron 
dispositivos más complejos, como pinzas que permitían manipular piezas o alcanzar objetos de díficil acceso facilitando el trabajo para el operario. En la era moderna, la teleoperación
ha estado evolucionando hasta el punto de incluir sistemas robóticos robustos que pueden ser controlados a distancia, permitiendo al operario poder realizar
tareas en entornos peligrosos e innacesibles para el ser humano como puede ser la exploración espacial, la medicina o la inspección nuclear. \newline

La intervencción del operador humano en los sistemas de teleoperación de robots es imprescendible, debe ser capaz de poder intepretar los datos sensoriales que proporciona el robot, así como de 
tomar decisiones robustas y precisas dependiendo de la situación. Esto conlleva tener una capacidad de realizar múltiple tareas simultáneamente adpatandose a situaciones imprevistas. \newline

Hoy en día, la teleoperación de robots tiene variedad de aplicaciones. Una de ellas puede ser la exploración espacial, en donde se utiliza la teleoperación
como técnica de manipulación remota como el Sojourner Rover. El Sojourner Rover\footnote{\url{https://www.astronomy.com/space-exploration/sojourner-nasas-first-mars-rover/}} 
es un pequeño robot móvil compuesto por 6 ruedas creado por los cientificos de la NASA para estudiar 
la superficie de Marte con la capacidad de enviar imagenes en directo y realizar analisis del terreno del planeta. Gracias a sus ruedas podía moverse por terrenos rocosos y de dificil acceso
ya que estaban equipadas materiales como de aluminio y acero inoxidable. \newline

\begin{figure} [H]
  \begin{center}
    \includegraphics[scale=0.4]{figs/introducción/Sojourner.jpg}
  \end{center}
  \caption{Sojourner Rover}
  \label{fig:Sojourner}
\end{figure}\

Con esta misión espacial se pudo probar como era el entorno marciano con técnicas realizadas en los laboratorios de la NASA demostrando que se podia realizar una teleoperación en 
el espacio abriendo el camino a futuros rovers como el Spirit, Opportunity y más \footnote{\url{https://spaceplace.nasa.gov/mars-spirit-opportunity/sp/}}. 

\subsubsection{Robótica Semiautónoma}
\label{sec:subseccion}
Los robots pueden realizar tareas de forma independiente siguiendo intrucciones preprogramadas o tomando decisiones en tiempo real, este enfoque se le conoce como autonomía o semi-autonomía, siendo la diferencia que en el enfoque semiautónomo todavía existe parte de teleoperación en el robot. Este enfoque permite que los robots puedan ser autónomos para poder
percibir su entorno y en la toma de decisiones pero con el handicap de que un operario humano puede controlarlo para poder ajustar parámetros, cambiar objetivos o intervenir 
en caso de emergencia. \newline

Aunque los robots semiautónomos puedan tomar decisiones en tiempo real, a menudo siguen instrucciones preprogramadas o reciben ordenes de un operario humano, esta toma de decisiones
puede incluir elegir la ruta más eficiente para navegar por un entorno peligroso como puede ser el robot submarino llamado Nereus. El Nereus\footnote{\url{https://www.bbc.com/mundo/ciencia_tecnologia/2009/06/090603_1541_nereus_robot_mar_mr}} 
es un vehículo submarino semiautónomo que puede ser manejado por control remoto que entro en servicio en el año 2009, su proposito fue explorar la Fosa de las Marianas, espeficamente el Abismo Challenger (es el punto más
profundo conocido en los océanos). Fue manejado meiante control remoto por pilotos que se encontraban en un barco en la superficie aunque el Nereus tambien podia cambiar al modo
de veículo autónomo pudiendo navegar libremente adaptandose a las condiciones del entorno sin intervencción humana directa. \newline

\begin{figure} [H]
  \begin{center}
    \includegraphics[scale=0.4]{figs/introducción/nereus.jpg}
  \end{center}
  \caption{Nereus}
  \label{fig:Nereus}
\end{figure}\

Lamentablemente, en 2014 durante una misión, el robot Nereus sufrió un colapso estructural y se perdio en el fondo del oceano. A pesa de esta perdida, los datos que se pudieron
recopilar en este robot submarino siguen siendo una fuente de conocimiento sobre las profundidades marinas. Este ejemplo de robot semiautónomo demuestra que se pueden realizar 
tareas en entornos peligrosos sin poner en riesgo la vida humana aunque tenga control por un operario\footnote{\url{https://www.elperiodico.com/es/ciencia/20140512/famoso-sumergible-nereus-pierde-fondo-mar-3271389}}. 

\newpage
\subsubsection{Robótica Autónoma}
\label{sec:subseccion}

La robótica autónoma consiste en tener robots que sean capaces de operar y realizar tareas de forma independiente sin la intervencción de un ser humano. En contraste con los 
robots teleoperados, este tipo de robots necesitan un comportamiento más robusto y preciso para realizar tareas independientes basandose en la percepción del entorno 
y en la toma de decisiones autónomas. \newline

El concepto de automía en los sistemas robóticos se esta convirtiendo en un area de investigación activa y en rápido desarrollo. 


Los avances en inteligencia artificial (IA), visión 
artificial, aprendizaje automático han facilitado la creación de robots autonómos capaces de llevar a acbo amplias variedades de tareas en entornos no estructurados y cambiantes \cite{upm70576}. \newline 

\begin{figure} [H]
  \begin{center}
    \includegraphics[scale=0.3]{figs/introducción/boston-dynamics-spot.jpg}
  \end{center}
  \caption{Spot de Boston Dynamics}
  \label{fig:Boston Dynamics}
\end{figure}\
\newpage
\section{Robótica aérea}
\label{sec:subseccion}

Dentro del campo de la robótica aérea tenemos los drones. Podemos definir un dron, como vehículo aéreo no tripulado (UAV), es un tipo de aeronave que puede operar sin la 
necesidad de un piloto humano a bordo. Estos dispositivos pueden ser controlados remotamente por un operador humano o navegar autonomamente incoporando software 
en su sistema. 
El origen de los drones se remonta a la Primera Guerra Mundial con el biplano Kettering Bug.
Este era un torpedo no tripulado de 240 kg (con una envergadura de 4,5 m, una longitud de
3,8 m y una altura de 2,3 m)\footnote{\url{https://www.nationalmuseum.af.mil/Visit/Museum-Exhibits/Fact-Sheets/Display/Article/198095/kettering-aerial-torpedo-bug/}} era propulsado por un motor alternativo. Podía volar de
forma autónoma hasta un punto específico, donde soltaba sus alas y caía en “caída libre”\footnote{\url{https://daytonunknown.com/2023/06/30/the-kettering-bug-the-worlds-first-drone/}}.
Avanzando en la historia, en 1935 se desarrolló el DH.82 Queen Bee\footnote{\url{https://dronewars.net/2014/10/06/rise-of-the-reapers-a-brief-history-of-drones/}}. Este era un blanco aéreo sin piloto que era controlado por radio. De hecho, parece que el término “dron” se originó a partir del nombre, que se refiere a la abeja macho que realiza un vuelo en busca de la abeja reina y luego fallece. \newline

Durante la Segunda Guerra Mundial, quizás el más conocido fue el V-1 "Flying Bomb"\footnote{\url{https://migflug.com/jetflights/the-v1-flying-bomb/}} , el primer misil
de crucero operativo del mundo, en donde su sistema de guía prestablecido incluía una brújula magnética que monitoreaba un autopiloto con giroscopios. También en este periodo, destacaremos el \textit{Proyect Aphrodite} \cite{Aphrodite}, fue un programa que tenía como objetivo convertir bombarderos en bombas voladoras no tripuladas que eran controladas por radio. Más adelante estos bombarderos no tripulados se utilizaron para volar a traves de nubes de hongo
después de las pruebas nucleares. \newline

Destacando más UAVs, tenemos la familia Teledyne Ryan Firebee/Firefly\footnote{\url{https://www.designation-systems.net/dusrm/m-34.html}}, estos sistemas generalmente se lanzaban 
desde el aire y se recuperaban mediante una combinación de paracaidas y helicopteros. El Lockheed D-21 fue uno de los sistemas más impresionantes durante la Guerra Fría. 
Este UAV fue propulsado por estatorreactor con velocidades mayores que Mach 3\footnote{\url{https://www.marchfield.org/aircraft/unmanned/d-21-drone-lockheed/}} . 
En la Edad Moderna, destacamos El Condor \cite{CondorUAV}, fue el primer UAS en utilizar navegación GPS y tecnología de aterrizaje automático y el Predactor\footnote{\url{https://www.airforce-technology.com/projects/predator-uav/?cf-view}}. 
En la época dorada, gracias a los avances anteriores se pudo desarrollar sistemas militares esenciales que han demostrado su valor y el desarrollo de vehículos aéreos no 
tripulados pequeños (small UAV). Este ultimo ha despertado un gran interés significativo resaltando como puntos de entrega al mercado civil ya que con sus cargas útiles
reducidas pueden ser portátiles y tener un coste menor . \newline

\begin{figure}[H]
  \begin{center}
    \subfigure[Kettering Bug]{
     \includegraphics[width=0.3\textwidth,height=0.2\textwidth ]{figs/introducción/historia_drones/kettering-bug.jpg}
     \label{f:Kettering Bug}}
    \subfigure[Queen Bee]{
     \includegraphics[width=0.3\textwidth,height=0.2\textwidth ]{figs/introducción/historia_drones/queen-bee.jpg}
     \label{f:Queen Bee}}
    \subfigure[V-1]{
      \includegraphics[width=0.3\textwidth,height=0.2\textwidth ]{figs/introducción/historia_drones/V-1.jpg}
      \label{f:V-1 "Flying Bomb"}}
    \subfigure[Proyect Aphrodite]{
      \includegraphics[width=0.3\textwidth,height=0.2\textwidth ]{figs/introducción/historia_drones/proyect-aphorite.jpg}
      \label{f:Proyect Aphrodite"}}
    \subfigure[Teledyne Ryan Firebee/Firefly]{
      \includegraphics[width=0.3\textwidth,height=0.2\textwidth ]{figs/introducción/historia_drones/Firebee.jpg}
      \label{f:Teledyne Ryan Firebee/Firefly"}}
    \subfigure[Lockheed D-21]{
      \includegraphics[width=0.3\textwidth,height=0.2\textwidth ]{figs/introducción/historia_drones/The_Lockheed_D-21.jpg}
      \label{f:Lockheed D-21"}}
    \subfigure[El Condor]{
      \includegraphics[width=0.3\textwidth,height=0.2\textwidth ]{figs/introducción/historia_drones/Boeing-Condor-UAV-23.png}
      \label{f:El Condor"}}
    \subfigure[Small UAV]{
      \includegraphics[width=0.3\textwidth,height=0.2\textwidth ]{figs/introducción/historia_drones/small-UAV.jpg}
      \label{f:small-UAV"}}
    \subfigure[Dron]{
      \includegraphics[width=0.3\textwidth,height=0.2\textwidth ]{figs/introducción/historia_drones/dron.jpg}
      \label{f:dron"}}
  \caption{Historia de los drones}
  \label{f:Drones}
  \end{center}
 \end{figure}

Cada vez es más común que los drones sean más sostificados y accesibles. Por ejemplo, el dron Ingenuity de la NASA se ha convertido en el primer vehículo aéreo autónomo en poder volar
sobre la superficie de otro planeta. Fue transportado a Marte mediante el rover Perseverance de la NASA, una vez fue posicionado el dron se elevó cerca de 3 metros realizando 
diferentes giros y desplazamientos tomando fotos a la superficie,teniendo la capacidad de escoger de forma autónoma los sitios de aterrizaje en el terreno marciano \footnote{\url{https://ciencia.nasa.gov/sistema-solar/finaliza-la-mision-del-helicoptero-ingenuity-en-marte/}}.
Este dron operaba de manera autónoma, controlado por sistemas de guía, navegación y control a bordo ejecutando los diferentes algoritmos desarrollados por la NASA. 

Uno de los grandes retos de este proyecto era demostrar la viabilidad del vuelo en la atmosfera de Marte, ya que su atmosfera esta compuesta por el 1\% de la densidad terrestre
dificultando el vuelo del dron. Sin embargo, gracias a su diseño ligero y a sus hélices especialmente diseñadas para crear suficiente sustentación en la atmósfera del planeta, el Ingenuity 
fue capaz de superar este desafío\footnote{\url{https://www.bbc.com/mundo/noticias-56738201}}. \newline
\begin{figure} [H]
  \begin{center}
    \includegraphics[scale=0.6]{figs/introducción/Ingenuity_II.jpeg}
  \end{center}
  \caption{El dron Ingenuity}
  \label{fig:Ingenuity}
\end{figure}\

Además, en su última fase, el Ingenuity realizó pruebas de vuelo experimentales para ampliar el conocimiento sobre cuáles eran sus límites aerodinámicos\footnote{\url{https://science.nasa.gov/mission/mars-2020-perseverance/ingenuity-mars-helicopter/}}.\newline

Otro ejemplo de uso de drones podemos tener control y mantenimiento de redes eléctricas y otras infaestructuras. Algunas construcciones constan de grandes alturas y tamaños, lo que puede
dificultar el trabajo y su correcto mantenimiento. No obstante, estas tareas con los drones se agilizan y se vuelven más eficientes y robustas, porque permiten poder
inspeccionar dichas infraestructuras desde cerca sin poner en peligro a la seguridad de los operarios. 
Hay drones que se encargarn en la monitorización de infraestructuras elétricas. \newline

Unión Fenosa, la distribución elétrica en España de Naturgy, en 2018 incorporó drones a sus instalaciones eléctricas para realizar labores de supervisión. Estos drones aportan 
soluciones optimizadas y eficientes en costes. Si tenemos en cuenta la longitud que puede tener las redes elétricas, el uso de estos vehículos
autónomos facilita las tareas de supervisión equipados de cámaras de última generación permitiendo al operario observar en tiempo real el estado de las infraestructuras. Además de que los
drones podrían acceder a zonas de difícil acceso para comprobar daños y poder repararlos \footnote{\url{https://www.ufd.es/blog/primer-vuelo-de-un-dron-mas-alla-de-la-linea-visual/}}. \newline 


\begin{figure} [H]
  \begin{center}
    \includegraphics[scale=0.5]{figs/introducción/drones-red-electrica.jpg}
  \end{center}
  \caption{Drones en inspección electrica en Galicia}
  \label{fig:Fenosa}
\end{figure}\
Asimismo, Amazon ha estado trabajando en el desarrollo de drones para la entrega de paquetes durante varios años denominado así Prime Air\cite{AmazonPrimeAir}, 
que consiste en un sistema de entrega de paquetes utilizando estos vehículos. Durante este programa, han realizado diferentes pruebas de reparto de paquetes a clientes
en 60 minutos o menos. Estas entregas se realizan desde sitios preparados con esta modalidad, el proceso de entrega de los drones empiezan en los centros de preparación de pedidos en 
donde los empleados seleccionarian el artículo, lo llevarían a la estación de embalaje y cuando este preparado su embalaje se deslizaría por una rampa hacia la fase de
entrega. 
En la fase de entrega los empleados preparan el paquete y la batería del dron para su destino. Cuando el dron halla realizado el despegue seguirá una ruta prestablecida con supervisión
de un operador para garantizar su entrega. \newline

Estos drones están automatizados para que logren volar a velocidades de 65 kilometros por hora, al llegar el dron a su detino, se dentendrá y descenderá lentamente para dejar el 
paquete en un area designada. Una vez el paquete halla sido entregado, el vehículo regresará al centro de preparación de pedidos para realizar la siguiente entrega 
\footnote{\url{https://logistica.cdecomunicacion.es/e-commerce/140453/prime-air-amazon-entregas-drones}} . 

\begin{figure} [H]
  \begin{center}
    \includegraphics[scale=0.2]{figs/introducción/dron-amazon.jpg}
  \end{center}
  \caption{El primer prototipo de dron de Prime Air}
  \label{fig:PrimerPrimeAir}
\end{figure}\

A lo largo de los años, Amazon ha seguido investigando y diseñando nuevos modelos de drones como el dron MK27-2\footnote{\url{https://www.europapress.es/portaltic/gadgets/noticia-amazon-prime-air-comienza-entregar-pedidos-drones-estados-unidos-20221229115034.html}}. Fue el primer 
dron que utilizó Amazon para 
las primeras entregas dentro del programa Prime Air durante el año 2023, se basaba en un dron eléctrico capaz de entregar paquetes con un peso máximo de 1,5 kilogramos a los clientes en menos de una
hora y capaz de realizar vuelos evitando obstáculos como puede ser las chimeneas o las torres de telefonía aunque no puede realizar entregas durante tormentas, vientos fuertes, temperaturas
extremas o cualquier situacion climatológica desfavorable. 

Este servicio solamente esta disponible para domicilios que tengan patios traseros que dispongan de espacio suficiente para que el dron pueda realizar el aterrizaje y la 
entrega del pedido.

\begin{figure} [H]
  \begin{center}
    \includegraphics[scale=1.7]{figs/introducción/MK27-2.jpg}
  \end{center}
  \caption{El dron MK27-2}
  \label{fig:MK27-2}
\end{figure}\

Sin embargo, Amazon Prime Air será más eficiente gracias al dron MK30 creado y diseñado por Amazon. Este pequeño dron será capaz de volar en diferentes condiciones climatológicas y 
constará de un sistema capaz de identificar y evitar obstáculos en el área de entrega. Será capaz de volar enn situaciones de lluvia ligera y poder realizar vuelos generando un 25\%
menos de ruido que el anterior modelo. Una novedad de este dron en comparación con los anteriores modelos es que será capaz de aterrizar en espacios más reducidos lo que conlleva a que
este tipo de servicio pueda llegar a más vencidarios. \newline
Se tiene previsto que se llegue a probar en el año 2024 empezando por cuidades como Texas y California en Estados Unidos
\footnote{\url{https://www.forbesargentina.com/innovacion/asi-nuevo-asombroso-dron-amazon-mk30-n42612}}. \newline


\begin{figure} [H]
  \begin{center}
    \includegraphics[scale=0.5]{figs/introducción/amazon-dron-mk30.jpg}
  \end{center}
  \caption{El dron MK30}
  \label{fig:MK30}
\end{figure}\

En un futuro cercano, puede que los drones sean más eficientes para las aplicaciones cíviles y científicas incluyendo protección contra incendios forestables, misiones agrícolas, 
ayuda en catastrofes y más. 
Las demostraciones actuales del uso de los drones han revelado el potencial que pueden tener pero aun así el acceso al espacio aéreo sigue siendo un factor limitante. Con el paso del 
tiempo, se irán desarollando nuevas tecnologías prácticas para poder permitir una integración segura en el espacio aéreo \cite{KrejciGarzon_2014}. \newline

En resumen, los drones son una tecnología emergente con un potencial significativo para transformar una variedad de industrias. Sin embargo, también plantean desafíos
únicos que deben ser abordados a medida que se integran más plenamente en nuestra sociedad. Con el desarrollo continuo de la tecnología de los drones y la evolución de las
regulaciones, es probable que veamos un aumento en la variedad de las aplicaciones de los drones en el futuro. \newline

\begin{figure} [H]
  \begin{center}
    \includegraphics[scale=0.3]{figs/introducción/drones-copilot.jpeg}
  \end{center}
  \caption{Ilustración de como serán los drones en un futuro creado con Copilot}
  \label{fig:Copilot}
\end{figure}\

\newpage
\section{La inteligencia artificial en la navegación autónoma de drones}
\label{sec:IA}

La incorporación de inteligencia artificial en el mundo de los drones desempeña un papel crucial en la navegación autonóma, permitiendoles tomar decisiones en tiempo real y adaptarse 
a entornos cambiantes de manera eficiente. Permitiendo a los drones poder aprender de sus experiencias y entender e interactuar con el entorno en el que se encuentran de una manera más
natural. \newline

Los drones que están equipados con IA pueden realizar vuelos de precisión, mantener la estabilidad incluso en condiciones adversas como fuertes vientos, y evitar obstáculos 
de forma dinámica. Esto es posible gracias a la combinación de datos sensoriales junto con los algoritmos de IA, lo que permite 
al dron interpretar su entorno y tomar decisiones en tiempo real.\newline 
Uno de los enfoques más destacadados en la navegación autonóma de drones es el aprendizaje automático. Este enfoque permite a los drones mejorar su objetivo a través de la experiencia 
y los datos recopilados durante el vuelo. \newline

Los algoritmos de aprendizaje automático, como las redes neuronales convolucionales (CNN), son utilizados para la detencción y clasificación
de objetos. Por ejemplo, las CNN son capaces de analizar imágenes capturadas por las cámaras a bordo del dron para identificar obstáculos, 
peatones o vehículos. Un tipo de aplicación de uso de redes neuronales puede estar en la detencción y clasificación de malas hierbas \cite{CSIC}. Mediante el sensor de la cámara, el dron es capaz de capturar imagenes en tiempo real 
para más adelante usar la red neuronal CNN YOLOv8\footnote{\url{https://github.com/ultralytics/ultralytics}} para detectar y clasificar las diferentes hierbas que puede haber en un campo de cultivo. Este tipo de aplicación es bastante útil para la inspección
agrícola ya que los drones pueden crear mapas detallados que permiten a los agricolas aplicar herbicidas de manera más eficiente y precisa, también este tipo de aplicación puede
ser útil para tener una monitorización general sobre la salud del cultivo. \newline 
Por otro lado, el reinforcement learning (RL) \cite{6025669} es una técnica dentro del aprendizaje automático que promete bastante en la navegación autonóma de drones. Esta metodología permite
a los drones aprendan a realizar trayectorias y planificar de rutas de forma autonóma, mejorando su desempeño a mediante un esquema de penalizaciones y recompensas permitiendo
así al dron poder tomar decisiones decisivas en situaciones puntuales. En este artículo 
\cite{ai2030023} precisamente se utiliza un algoritmo de RL para la evitación de obstáculos en un espacio continuo y se llega a conseguir que con estos
tipos de algoritmos que un dron pueda llegar aprender comportamientos y tomar decisiones por él mismo. 

\begin{figure} [H]
  \begin{center}
    \includegraphics[scale=0.6]{figs/Diseño/RL/Esquema.jpeg}
  \end{center}
  \caption{Esquema de Reinforcement Learning}
  \label{fig:Reinforcement Learning}
\end{figure}\



En conclusión, la inteligencia artificial puede ser fundamental en la navegación autonoma de drones al permitirles percibir su entorno, podemos tomar decisiones y planificar acciones 
de manera anticipada y autonóma. A medida que vayamos avanzando, se espera que los drones tengan más sistemas de inteligencia artificial abordo para cubrir una amplia gama de tareas
de manera autónoma, lo que abriría nuevas fronteras en campos como el rescate, la vigilancia, la logística y la exploración, y que promete seguir transformando la forma en que 
interactuamos con el espacio aéreo en un futuro. 

\newpage
\section{Navegación autónoma en Airsim basada en inteligencia artificial y aprendizaje por refuerzo}
\label{sec:Navegación autónoma}

En este trabajo realizaremos una navegación autonóma en entornos de simulación por Airsim mediante redes neuronales, algoritmos de aprendizaje automático y aprendizaje 
por refuerzo. Demostrando así que un dron es capaz de tomar decisiones por si mismo dependiente de la situación en la que se pueda encontrar en un entorno de carreteras.

