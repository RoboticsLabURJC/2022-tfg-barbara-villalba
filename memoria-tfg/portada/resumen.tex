\cleardoublepage

\chapter*{Resumen\markboth{Resumen}{Resumen}}

Una de las principales áreas de estudio de la robótica es la navegación autónoma. En la actualidad, esta se lleva a cabo mediante dos tipos de algoritmos, los que se podrían llamar clásicos que se 
basan en la planificación de rutas y trayectorias mediante algoritmos clásicos, o cada vez tomando más protagonismo aquellos basados en Inteligencia Artificial. Dentro de la robótica aérea se están 
creando una gran número de aplicaciones variadas gracias a la capacidad de los drones de volar, y tener una perspectiva del entorno y alcanzar zonas a los que un robot con ruedas no podría llegar.

En este TFG se explora un método de navegación basado en Inteligencia Artificial, en el que se procura que el dron se mueva dentro de un área determinada, en este caso un carril de una carretera.
Este tipo de navegación, podría tiene muchas aplicaciones, por ejemplo, la inspección de carreteras a gran altura para analizar el estado de las mismas.

Para conseguirlo se han usado técnicas de Deep Learning, como las Redes Neuronales de segmentación para la detección del carril y extracción de sus características, así como otras técnicas clásicas para mejorar la información extraída.
Así como algoritmos de Reinforcement Learning, en este caso, Q-Learning, para que el sistema de control del dron fuese capaz de aprender las acciones a realizar en función de la información sensorial obtenida.

Todo ello se ha realizado empleando el simulador AirSim, en el que se simula un entorno realista, que el dron deberá completar la tarea anteriormente descrita. Además de emplear el middleware robótico ROS
para las comunicaciones entre el simulador, y el software implementado en esta solución.
