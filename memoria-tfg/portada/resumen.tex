\cleardoublepage

\chapter*{Resumen\markboth{Resumen}{Resumen}}

Dentro del mundo de la róbotica, la navegación autónoma emerge una de las áreas más emocionantes especialmente la navegación autónoma de drones con sistemas de inteligencia artificial. 
Esto permite que los vehículos aéreos no tripulados no solo ejecuten tareas preprogramadas, sino que también sean capaces de aprender la adaptación de entornos dinámicos y cambiantes.
Los drones se han convertido en un gran desafio en el mundo de la robótica aérea.Pueden ser programados para realizar tareas específicas, como mapear tarrenos en aplicaciones 
cartográficas o entregar suministros médicos en zonas de díficil acceso. Su versatilidad y capacidad de operar de manera autónoma convierte a estos robots aéreos en herramientas 
valiosas en diversas aplicaciones.\newline

Además de la navegación autónoma, la inteligencia artificial permite a los drones,por ejemplo, ser entrenados para reconocer patrones y objetos en su entorno permitiendoles realizar
tareas como la identificación de personas en situaciones de busqueda y rescate o la detencción de anomalías en infraestructuras con algoritmos de aprendizaje automático. Sin embargo, a pesar
de estos avances, la navegación autónoma e inteligencia artificial en drones sigue siendo un área de investigación debido a la necesidad de algoritmos de aprendizaje más robustos 
que puedan cumplir con éxito en un futuro próximo.\newline

Con este Trabajo de Fin de Grado se demuestra que la navegación autónoma de drones es capaz de tener un comportamiento autónomo en entornos realistas y complejos de carreteras tomando decisiones
en tiempo real para alcanzar sus objetivos de manera eficiente y segura. Para poder lograr esto, se explorarán y se implementarán técnicas de algoritmos de aprendizaje autómatico y de inteligencia
artificial con un enfoque en particular el aprendizaje por refuerzo utilizando entornos de simulación de Airsim. 





