\cleardoublepage

\chapter*{Resumen\markboth{Resumen}{Resumen}}

La revolución tecnológica ha tenido un impacto sin precedentes en el mundo de la róbotica, desde los robots industriales en las fábricas hasta tener robots capaces de entablar conversaciones
con personas. Dentro del mundo de la róbotica, la navegación autónoma emerge una de las áreas más emocionantes como la navegación autonoma de drones con sistemas de inteligencia artificial, permitiendo 
permitiendo que los vehículos aéreos no tripulados no solo ejecuten tareas preprogramadas, sino que también sean capaces de aprender la adaptación de entornos dinámicos y cambiantes.
Convirtiendo a estos pequeños vehículos en un gran desafío en el mundo de la robótica aérea. Los drones pueden ser programados para realizar tareas específicas por ejemplo mapear terrenos en aplicaciones
cartográficas o entregar suministros médicos en zonas de dícifil acceso.
 \newline

A parte de la navegación autónoma, la inteligencia artificial permite a los drones por ejemplo ser entrenados para reconocer patrones y objetos en su entorno lo que les puede permitir realizar
tareas como la identificación de personas en situaciones de busqueda y rescate o la detencción de anomalías en infraestructuras con algoritmos de aprendizaje automático. Sin embargo, a pesar
de estos avances, la navegación autónoma e inteligencia artificial en drones sigue siendo un área de investigación debido a la necesidad de algoritmos de aprendizaje más robustos 
que en un futuro próximo se podrá llegar a cumplir con éxito. \newline

Con este Trabajo de Fin de Grado demostraremos que la navegación autónoma de drones es capaz de tener un comportamiento aútomo en entornos realistas y complejos de carreteras tomando decisiones
en tiempo real para alcanzar sus objetivos de manera eficiente y segura. Para poder lograr esto, se explorarán y se implementarán técnicas de algoritmos de aprendizaje autómatico y de inteligencia
artificial con un enfoque en particular en el aprendizaje por refuerzo utilizando entornos de simulación de Airsim. 





